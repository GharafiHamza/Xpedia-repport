\documentclass[12pt]{report}

\setcounter{tocdepth}{3}
\setcounter{secnumdepth}{3}

\usepackage[utf8]{inputenc}
\usepackage[french]{babel}
\usepackage[margin=1in]{geometry}
\usepackage{amsmath}
\usepackage{hyperref}
\hypersetup{
    colorlinks,
    linkcolor=black
}
\usepackage[nameinlink,noabbrev]{cleveref}
\usepackage{graphicx}
\usepackage{tocloft}
\usepackage{multicol}
\usepackage{enumitem}
\usepackage{float}
\usepackage{tabularx}
\usepackage{ragged2e}
\usepackage[table]{xcolor}
\usepackage{xurl}


\title{L'apport de l'inteligence artificielle sur les clichés de radiographie pédiatrique
\\ {\Large Ecole National des Sciences Appliquées}}

\author{GHARAFI Hamza}
\date{2021/2022}
\graphicspath{ {./resources/} }

\begin{document}

    \newcommand{\listequationsname}{Liste des Equations}
    \newlistof{myequations}{equ}{\listequationsname}
    \newcommand{\myequations}[1]{%
    \addcontentsline{equ}{myequations}{\protect\numberline{\theequation}#1}\par}

    \chapter*{\centering Résumé}
    Dans ce rapport nous abordons la question de l'apport de l'intelligence artificielle à l'interprétation des images radiographiques thoraciques, en particulier pour une population pédiatrique sur un certain nombre de pathologies.

    L'analyse de l'existant pose un défi récurrent en la matière, à savoir le manque de données, tant en quantité qu'en nature. Notre projet porte donc sur l'enrichissement de la base de données par la collecte de données et par le développement d'une stratégie d'apprentissage qui résout ce genre de problème. Pour ce faire nous avons développé une application web de collecte de données médicales à savoir les images radiographiques et les rapports associés tout en préservant l'anonymat des patients concernés, puis nous avons procédé à la préparation de ces données pour constituer un jeu de données adapté aux séances d'apprentissage.
    Dans un troisième temps, des modèles paramétrés ont été construits pour lancer un plan d'expériences afin d'optimiser le modèle en termes de précision et de performance. 
    
    Les premiers résultats affirment la viabilité de la méthode, mais suggèrent l'adaptation des hyperparamètres du modèle afin de réduire le phénomène de suradaptation et ils proposent également la subdivision de notre problème en sous-tâches d'apprentissage par pathologie afin de réduire la complexité et la dimensionnalité du problème.


    \vspace*{2cm}
    \textbf{Mots clés:} Intelligence artificielle, Apprentissage profond, Radiographie pulmonaire, Médecine, Radiologie, Pathologie, Pédiatrie.

    \chapter*{\centering Abstract}
    In this report we deal with the question of the contribution of artificial intelligence to the interpretation of thoracic X-ray images, in particular for a pediatric population on a number of pathologies.
    
    The analysis of what already exists raises a recurring challenge in this subject, namely the lack of data, both in quantity and in nature. Our project therefore focuses on the enrichment of the database by means of data collection and by the development of a learning strategy that solves this kind of problem. 
    To do this we have developed a web application for the collection of medical data namely x-ray images and associated reports while maintaining the anonymity of the patients concerned, then we proceeded to the preparation of these data to build a data set suitable for learning sessions. 
    Thirdly, parameterized models were built to launch a design of experiments in order to optimize the model in terms of precision and performance. 
    
    The first results affirm the viability of the method, but suggest the adaptation of the hyperparameters of the model in order to reduce the phenomenon of over-adaptation and they also propose the subdivision of our problem into learning sub-tasks by pathology in order to reduce the complexity and the dimensionality of the problem.


    \vspace*{2cm}
    \textbf{Keywords:} Artificial intelligence, Deep learning, Chest x rays, Medicine, Radiology, Pathology, Pediatric.

    \chapter*{\centering Remerciment}
    \centering
        \subsubsection*{A ma maman, EL MORTAQY Rachida}
        Tous les mots de la terre ne seront pas suffisants pour exprimer ce que je ressens envers toi. Tu m’as tout donné, tout transmis, et tu as sacrifié ta vie pour que je sois là où je suis aujourd’hui. Tu es ma mère, ma meilleure amie, ma sœur et ma confidente. Tu es ma boussole, qui m’a toujours guidé dans mes moments de doute et de faiblesse. Tes yeux, remplis d’amour et de courage, m’ont élevé au-dessus de tous les obstacles de la vie. Le sens de la responsabilité et la détermination que tu m’as inculquée seront toujours en moi. Tu ne peux savoir la joie que je ressens lorsqu’on me félicite de l’éducation que j’ai reçue et des qualités que tu as su cultiver en moi. Je suis fier d’avoir eu une femme, ambitieuse, généreuse, cultivée, joyeuse, qui a tant donné pour des générations d’élèves qui viennent te remercier de ce que tu leur as transmis. Je suis chanceux d’avoir une personne comme toi à mes côtés, et je remercie dieu pour cela à chaque instant.

        Comme nous l’avons toujours dit, nous ne sommes pas que mère et fils, nous sommes partenaires, et cette réussite est autant la mienne que la tienne. Puisse Dieu te garder auprès de moi, en bonne santé, aussi longtemps que possible. Je t’aime.
        \subsubsection*{A mon père, GHARAFI Aissa }
        Long a été le chemin, nombreuses ont été les contraintes. Mais tu as su tenir tête, et te sacrifier, depuis le jour où je suis né. Comblé est le mot, quand je repense à l’amour que tu me portes, et qui a bercé mon enfance. Ta sagesse, ton sang-froid et ta patience, font de moi ce que je suis aujourd’hui. Tu as su créer la balance, au côté de maman, afin que je puisse recevoir la meilleure éducation possible. J’espère te rendre fier papa, et honorer ton nom. Je t’aime.
        \subsubsection*{A Notre Encadrant Du PFE Et Juge De La Soutenance Professeur MASSAQ Abdellah, Professeur A L'Ecole National Des Sciences Appliquées }
        C’est un réel honneur que vous acceptez d'encadrer mon travail de projet de fin d'étude, Votre disponibilité,sympathie ainsi que votre esprit

        cartésien a été une véritable source d'inspiration tout au long de mon projet de fin d'étude. Notre école vous doit énormément et nous sommes fiers d’avoir un professeur comme vous. Veuillez trouver ici le témoignage de mon plus grand respect.
        \subsubsection*{A Nos Encadrant Du PFE Professeur JALAL Hicham, Professeur De Radiologie Et Chef De Service De Radiologie, A Hôpital Universitaire Mohammed VI De Marrakech }
        Permettez-moi de vous remercier du fin fond de mon cœur, pour la confiance que vous m’avez accordée, en me donnant à traiter un sujet aussi original. Travaillé sous votre direction était un réel honneur.Votre sérieux, vos qualités pédagogiques et votre intarissable bonté m’inspirent beaucoup d’admiration et de respect. Veuillez trouver ici, cher Professeur, le témoignage de ma profonde reconnaissance et de mon grand respect.

        \subsubsection*{A Nos Maitres Et Juges De La Soutenance Professeurs ATLAS Abdelghafour et OUMOUN Mohamed, Professeur A L'école Ntionale Des Sciences Appliquées De Marrakech}
        Je vous remercie de la spontanéité et l'extrême gentillesse avec laquelle vous avez bien voulu accepter de juger ce travail. Veuillez trouver ici, cher Professeur, le témoignage de ma profonde reconnaissance et de mon grand respect.
        \subsubsection*{A toute la famille GHARAFI et EL MORTAQY, }
        J’ai eu une chance inestimable d’être né dans deux familles aussi aimantes, généreuses et soudées. Je me suis toujours senti bercé par votre amour, porté par vos encouragements, et confiant par la sécurité que vous m’avez conférée.J’espère rendre fiers mes arrière-grands-parents, comme ils continuent à nous rendre tous fiers, des siècles après. 
        \subsubsection*{Aux précieux amis rencontrésà l'ENSA', entre autres : LADRAA Adelhak, ESSERHIRE Mohamed, CHAIB Moncif , … }
        Écrire mes sentiments pour chacun de vous me demandera surement plusieurs thèses. Je vous considère tous, sans exception, comme mes amis et frères. Nous avons partagé des moments qui m’ont laissé me rapprocher de vous. Vous êtes tous des personnes honnêtes, généreuses, bienveillantes et loyales, et c’est pour cela que vous avez une place particulière dans ma vie. En étant parmi vous, tout mon parcours en médecine n’a été que joie et bonheur. Je serai toujours là pour vous, car vous êtes et seriez toujours là pour vous. Je vous aime. 
        
    \justifying
       
        
    
    \clearpage
    \tableofcontents
    \clearpage
    \listoffigures
    \clearpage
    \listoftables
    \clearpage
    
   \listofmyequations


   \chapter*{Introduction générale}
   Aujourd'hui dans la plupart des services de radiologie la demande croissante de divers types d'examens radiologiques par différents spécialistes a rendu le nombre d'images destinées à l'interprétation si énorme qu'il dépasse largement les capacités des radiologues malgré les efforts fournis quotidiennement   même avec l'augmentation du nombre de radiologues employés; Chose qui a fait l’objet d’une recherche de solutions facilitant ou proposant des interprétations complètes ou partielles des images radiologiques afin de lutter contre la surcharge de travail chez les radiologues.
 
    L'émergence de l’intelligence artificielle et l’évolution rapide des technologies informatiques traitant les mégadonnées ou les données massives (Big Data) ont ouvert la voie au développement de systèmes informatiques simulant une ou des parties d’interprétation de diverses données médicales et notamment des images radiologiques.
    
    Cela dit, même si l'adoption du terme intelligence artificielle est relativement récente, l'utilisation de cette technologie ne l'est pas ; notamment en radiologie médicale et plus particulièrement son utilisation pour la détection automatique de certaines anomalies sur les images tomodensitométriques : la détection des micronodules sur les TDM thoraciques par exemple.

    Les algorithmes d’apprentissage adoptées par l’IA actuellement sont devenues miraculeux avec des records de performances parfois très puissants qu’ils dépassent même ceux des radiologues, notamment dans le cas des images radiologiques peu contrastées et dans la détection de très petites anomalies; 

    Cette remarquable révolution d’IA pose la question de la possibilité de sa participation totale ou en partie au flux de travail des services de radiologie en particulier dans la tâche d'interprétation ou bien de lecture des images radiologiques qui était jusqu'à présent considérée comme une tâche exclusive des radiologues. , cette question à son tour relève une problématique aussi complexe et aussi ancienne que la première interaction homme-machine et qui questionne simplement la possibilité du remplacement de l’homme par la machine , la réponse par oui  est tout à fait valide , possible et légitime dans plusieurs domaines sauf que dans ce cas-là vu la complexité ,la sensibilité et la vitalité du domaine de son application la réponse à notre problématique prend plusieurs dimensions qu’on va traiter ultérieurement dans le chapitre \ref{general_project_context}.

    Jusqu'à Aujourd'hui de nombreuses institutions de santé exploitent des technologies de l’IA qui ont réussi à prouver leur efficacité et à posséder des licences de la part des autorités de santé pour leur utilisation dans l'amélioration de la qualité des images radiologiques et leur manipulation mais la chose la plus intéressante c'est leur obtention récente d’une licence pour le triage des images de radiographie thoracique normales; Cette dernière est considérée comme un grand pas donnant plus de fiabilité et de confiance à l’IA  vu son implication directe dans la prise de décision diagnostic même avec la nécessité d’une supervision humaine des résultats.

    Cette confiance accordée actuellement à l’Ia ne traduit pas une tendance vers le remplacement du rôle des médecins radiologues mais plutôt une assistance et un coup de main offert en sort d’outils façonnent et soutiennent le rôle du  radiologue par aide à : la génération automatique des rapports , la réduction de temps des interprétations , la formation et l’assistance des juniors radiologues et des médecins moins expérimentés , et à assister à la détection des lésions ou autres anomalies radiologiques sur des images moins contrastées qu’un radiologue humain pourrait ne pas déceler. 


    \chapter{Contexte générale du projet}

    \section*{Introduction}
        Dans ce chapitre, nous allons mettre notre projet en contexte, c'est-à-dire definit l'organisme d'acceuil, problématique, état d'art et finalement solution proposée, donc en résumé, notre travail dans ce projet est une etude des radiographies médicales plus clairement la classification de ces derniers suivant une list de diagnostiques, mais plus important encore, notre objectif est de répondre à la question "pouvons-nous obtenir de bons résultats à partir d'une petite base de donées?".

    \section{Organisme d'acceuil}

        \subsection{CHU Mohammed VI de Marrakech}
            \begin{figure}[h]
                \centering
                \includegraphics[width=0.3\textwidth]{icon_chu.png}\label{fig:chu}
                \caption{Icone du CHU}
            \end{figure}
            Le Centre Hospitalier Universitaire Mohammed VI de Marrakech joue un rôle  majeur et important dans l’offre de soins non seulement dans la région de Marrakech-Safi, mais  dans toute la partie sud du Royaume.

            Il se compose de quatre hôpitaux et deux centres, d’une capacité de 1548 lits dont :
            \begin{enumerate}
                    \item L’Hôpital IBN TOFAIL à vocation médico-chirurgicales d’une capacité de 409 lits.
                    \item L’Hôpital MERE – Enfant à vocation gynéco-obstétricale et pédiatrique d’une capacité de 247 lits.
                    \item L’Hôpital IBN NAFIS à vocation psychiatrique d’une capacité de 220 lits.
                    \item L’Hôpital AR-RAZI à vocation médico-chirurgicales d’une capacité de 586 lits.
                    \item Le Centre d'hématologie-Oncologie : 86 lits.
                    \item Le Centre de recherche clinique.
            \end{enumerate}

            \subsubsection{Chiffres clés de production e CHU}
            Le Centre Hospitalier Mohammed VI est un établissement public doté de la personnalité morale et de l’autonomie financière. Il est soumis à la tutelle du Ministère de la Santé. Il a été crée en vertu de la Loi 82.00 promulguée par le Dahir 1.01.206 du 10 Joumada II 1422 (30 août 2001) modifiant et complétant la loi 37.80 relative aux centres hospitaliers, promulguée par le Dahir 1.82.5 du 30 rabia I (15 janvier 1983).

        
            \subsubsection{Service de radiologie mère-enfant}\label{service_radio}
            Notre travail a été mené au sein du service de radiologie de l’hopitale mère-enfant , qui se spécialise dans l’imagerie médicale pédiatrique et gynécologique , avec un équipement qui comprend la majorité des diverses outils d’imagerie médicale.
            
            \vspace{3mm}
            \begin{multicols}{2}
                \begin{itemize}
                    \item[$\bullet$] Scanners.
                    \item[$\bullet$] Appareil de radiologie télécommandée.
                    \item[$\bullet$] Appareils de radiologie conventionnelle.
                    \item[$\bullet$] Amplificateurs de brillance.
                    \item[$\bullet$] Echo-cardiographies.
                    \item[$\bullet$] Echo-doppler couleur.
                    \item[$\bullet$] Echographes.
                    \item[$\bullet$] Ostéodensitométre.
                    \item[$\bullet$] Appareils de radiologie mobile.
                    \item[$\bullet$] Mammographes.
                    \item[$\bullet$] Systèmes de numérisation par écran ERLM.
                \end{itemize} 
            \end{multicols} 
            \vspace{3mm}

        \subsection{Collaborateurs}\label{Collaborateurs}
            \begin{itemize}[label=$\bullet$]
                \item Professeur et chef de service de radiologie médicale mère-enfant:\newline
                Superviser les autres collaborateurs chargés de la collecte de données et suivre l'avancement du projet étape par étape.
                \item Thésard de médecine:\newline
                Collection de radiographies pulmonaires pédiatriques.
                \item 8 Résidents médecins radiologue seniors et juniors:\newline
                4 résidents juniors\newline
                4 résidents séniors \newline
                Diagnostic et remplissage des données concernant les clichés déjà collectés.
                \item Techniciens radiologues:\newline
                Fournir les radiographies d'intérêt pour notre projet.
            \end{itemize}


    \section{Contexte générale du projet}

        \subsection{Problématique}
            La radiologie médicale est considérée comme le domaine de la médecine qui a le plus bénéficié de l'IA  vue la diversité des zones de son application, voici une liste non exhaustive de ces applications:
            \vspace{3mm}
            \begin{itemize}
                \item[$\bullet$] automatisation de la détection des images pathologiques.
                \item[$\bullet$] la détection des lésions incidentes, non recherchées « a priori ».
                \item[$\bullet$] fiabilisation de l’interprétation des images.
                \item[$\bullet$] identification des motifs, autorisant la classification de lésions.
                \item[$\bullet$] établissement des comptes rendus uniformisés.
                \item[$\bullet$] traitement de larges cohortes d’images radiologiques. 
            \end{itemize}
            \vspace{5mm}

            Dans notre étude on va se concentrer sur l’automatisation de détection d’anomalies ou de constatations radiologiques sur des clichés numérisés de radiographie thoracique standard chez la population pédiatrique.

            La plupart des études d'IA pour l'interprétation des radiographies thoraciques sont appliquées sur des images de population adulte ( tranche d'âge supérieure à 18 ans) avec une couverture de la vaste majorité des anomalies radiologiques et des pathologies thoraco-pulmonaires.

            Alors qu’au niveau pédiatrique, ils se sont concentrés sur un nombre limité de pathologies ou d'anomalies radiologiques sans faire de progrès significatifs en raison de l'absence de  jeu de données  pédiatriques à grande échelle.

            \subsubsection{Procédure de détection des anomalies par les radiologues sur les radiographies thoraciques}
            \paragraph*{Rappel :Formation de l’image radiologique}
            \begin{enumerate}
                \item Tube de Coolidge produit un faisceau de RX
                \item Faisceau incident et homogène de RX
                \item Patient atténuant le faisceau de RX
                \item Faisceau sortant (transmis) de RX hétérogène: image radiante
                \item Appareil de détection reçoit le faisceau transmis
            \end{enumerate}
            La radiographie thoracique est un outil très performant en médecine clinique et savoir interpréter correctement une radiographie thoracique est indispensable aux cliniciens.
            \begin{multicols}{2}
                \begin{itemize}[label=$\bullet$]
                    \item Cadre osseux
                    \item Parties molles
                    \item Diaphragme
                    \item Coeur
                    \item Médiastin
                    \item Hiles
                    \item Poumon
                \end{itemize}
            \end{multicols}
            \begin{figure}[h]
                \centering
                \includegraphics[width=0.4\textwidth]{thorax_osseux.png}
                \caption{Thorax osseux. a. Arc postérieur de la côte; b. arc moyen de la côte ; c. arc antérieur de la côte ; d. extrémité antérieure de la côte; e. clavicule vue en enfilade avec nouure centrale (bras relevés); f. omoplate; g. épine de l'omoplate; h. métaphyse humérale ; i. cartilage de conjugaison; j. épiphyse humérale ; k. cavité glénoïde; l. corps vertébral; m. pédicules; n. disque intervertébral.
                }\label{fig:thorax}
            \end{figure}
           
            \begin{figure}[H]
                \centering
                \includegraphics[width=0.6\textwidth]{dia_mediastinal.png}
                \caption{Diagramme des structures médiastinales à analyser sur une Radiographie thoracique PA}\label{fig:mediastinal}
            \end{figure}
            \subsubsection{Les differences radio anatomiques entre les clichés pédiatriques et adultes}

            Sept différences entre la radiographie thoracique d'un nourrisson et celle d'un adulte.

            \begin{figure}[H]
                \centering
                \includegraphics[width=0.6\textwidth]{differences.png}
                \caption{Les différences radio-anatomiques entre l’adult et l’enfant}\label{fig:differences}
            \end{figure}

            \begin{enumerate}
                \item présence du thymus:
                \begin{figure}[H]
                    \centering
                    \includegraphics[width=0.6\textwidth]{thymus_enfant.png}
                    \caption{Les variations de l’image du thymus chez l’enfant}\label{fig:thymus}
                \end{figure}
                Aspects radiographiques normaux du thymus : aspect ondulé des bords du thymus sur un cliché réalisé en expiration (flèche noire) (A),aspect en voile latine (flèche blanche) (B), extension du thymus jusqu'à la coupole diaphragmatique (flèche noire en pointillés) (C),fausse impression de cardiomégalie du fait de l'extension inférieure du thymus (flèches blanches en pointillés) (D).	
                \item non visibilité du crosse aortique.
                \item portion antérieure du gril costal entièrement cartilagineuse, donc non visible sur la radiographie.
                \item courbure claviculaire accentuée du fait de la position des bras au-dessus de la tête.
                \item point d'ossification huméral supérieur.
                \item Déviation trachéale:
                \begin{figure}[H]
                    \centering
                    \includegraphics[width=0.6\textwidth]{tracheal.png}
                    \caption{La deviation trachéale}\label{fig:tracheal}
                \end{figure}
                Déviation trachéale physiologique vers la droite chez un nourrisson de 20 mois sur une radiographie de thorax réalisée en expiration (flèche noire) (A). Sur le cliché en inspiration, la trachée redevient rectiligne (flèche blanche) (B).
                \item L'incidence:\newline
                L'incidence de face en inspiration est suffisante dans la majorité des cas.
                Chez le petit enfant, elle est réalisée en incidence antéro-postérieure, puis en incidence postéro-antérieure lorsque l'enfant devient coopérant (après 4 ans environ).
                Chez le nouveau-né et le petit nourrisson ne tenant pas assis, l'examen est réalisé en décubitus dorsal.

            \end{enumerate}

            \subsubsection{La partie informatique de la problématique}\label{partie_info}
                Dans notre projet on veut crée un produit qui pourrait atténuer le flot des clichés thoraco pulmonaire qui doivent être traités individuellement par les radiologues, et en particuliérement les radiographies pédiatriques qui n'ont pas assés d'attention que les radiographies adulte, mais suivant une liste des anomalies predéfinie ou plutôt imposé par les base données publique à l'utilisation des étudiants, la liste des anomalies et la suivante:

                Dans notre projet, nous voulons créer un produit capable de réduire le flux de radiographies thoraco-pulmonaires, que les radiologues doivent traiter séparément. Anomalies, listes d'anomalies mandatées ou imposées par les bases de données publiques à l'usage des étudiants,  et ce qui suit:
                \begin{multicols}{2}\label{list_dia}
                    \begin{enumerate}
                        \item Aucun résultat
                        \item Élargissement cardio-médiastinal
                        \item Cardiomégalie
                        \item Opacité pulmonaire
                        \item Lésion pulmonaire
                        \item Œdème
                        \item Consolidation
                        \item Pneumonie
                        \item Atélectasie
                        \item Pneumothorax
                        \item Épanchement pleural
                        \item Autre lésions pleurales 
                        \item Fracture
                        \item Appareils de soutien
                    \end{enumerate}
                \end{multicols}

                Pour plus de detail sur la liste des anomalies voir la sous-section \ref{chexpertDB}.

        \subsection{Etat d'art}
            \subsubsection{Modèles de détection automatiques de pathologies thoraciques précise chez l‘adulte}
            La dernière décennie a vu la réalisation de plusieurs applications de l'IA sur la radiographie standard et plus particulièrement la radiographie thoracique, notamment chez l'adulte.
            Ces applications ont inclus 
            
            soit la détection automatique de motifs en relation avec des anomalies thoraco-pulmonaires par exemples les études suivantes:
            
            soit la détection de pathologies thoraco-pulmonaires précises telles que:
            
            \begin{itemize}[label=$\bullet$]
                \item La détection des lésions tuberculeuses pulmonaires
                \hspace*{1cm}Les algorithmes d'IA peuvent être des outils de triage très précis et utiles pour la détection de la tuberculose dans les régions à forte prévalence.
                \item La détection précoce des lésions du cancer du poumon
                \hspace*{1cm}L'algorithme d'IA peut améliorer les performances des lecteurs pour la détection des cancers du poumon sur les radiographies pulmonaires lorsqu'il est utilisé comme second lecteur.
                \item Récemment le diagnostic du SDRA et pronostic des patients atteints de COVID-19
                \hspace*{1cm}Schéma d'évaluation par radiographie pulmonaire assisté par intelligence artificielle pour COVID-19
            \end{itemize}

            \subsubsection{Modèles de détection de Pathologies précises ou d‘anomalies à la rx thoracique chez la Population Pédiatrique}

                Sur 29 ensembles de données de radiographie pulmonaire accessibles au public, 2 ensembles de données ne comprenaient que des radiographies pulmonaires pédiatriques et 7 ensembles de données comprenaient à la fois des patients pédiatriques et adultes. 

                Padash et al.ont identifié 55 articles mettant en œuvre un modèle d'IA pour l'interprétation des radiographies thoraciques pédiatriques ou des radiographies thoraciques pédiatriques et adultes. La classification des radiographies pulmonaires comme pneumonie était l'application la plus courante de l'IA, évaluée dans 65 \% des études. Bien que de nombreuses études rapportent une précision diagnostique élevée, la plupart des algorithmes n'ont pas été validés sur des ensembles de données externes.

        \subsection{Solution Proposée}\label{Solution_prop}
            Les  données de la population pédiatrique sont beaucoup moins documentées que celles de la population adulte, et notamment pour les radiographies thoraciques, ce qui nous inspirer pour créer ou collecter notre propre base de données, toujours en respectant la liste des anomalies déjà citées dans la sous-section \ref{list_dia}.
            La solution proposée se compose essentiellement de trois unités principales qui sont bien connectées les unes aux autres pour former un flux de travail fluide et solide.
            \begin{figure}[H]
                \centering
                \includegraphics[width=1\textwidth]{sol_prop.jpg}
                \caption{Architecture générale du solution proposé}\label{fig:sol_prop}
            \end{figure}
            \subsubsection*{Application Web}
                Premiére phase:
                Une application web qui peut collecter des données. La collecte est effectuée par les collaborateurs voir sous-section \ref{Collaborateurs}. L'application doit être simple, légère, rapide et claire avec une interface concise direct afin de ne pas prendre beaucoup de temps au médecin.
            \subsubsection*{Pipline de données}
                Deuxiéme phase:
                Étant donné que le pipeline de données sera dans une phase de pré-apprentissage, nous devons préparer une base de données bien structurée et propre où toutes les valeurs sont définies et non nulles.
            \subsubsection*{Les modeles d'apprentissage en profondeur}
                Troisiéme phase:
                Construisez des modèles d'apprentissage en profondeur qui peuvent pré-diagnostiquer les anomalies dans le thorax osseux à partio des clichés thoraciques pédiatrique.


    \section*{Conclusion}
    En conclusion, voici quelques réflexions générales sur le projet et sur ce que vous devez faire pour pouvoir créer un projet capable d'effectuer  des pré-diagnostiques des radiographies pulmonaires pédiatriques.
    \begin{enumerate}
        \item Création d'une base de données pédiatrique étiquetée suivant la liste \ref{list_dia}
        \item Création d'un pipline de données pour prétraitement des données
        \item Création des modèles d'apprentissage en profondeur
    \end{enumerate}
    Cependant, afin de poursuivre le projet, nous devons d'abord savoir ce qu'est  l'apprentissage en profondeur.\label{general_project_context}
    \chapter{Apprentissage profond}

\section*{Introduction}
    Dans ce chapitre, je vais vous présenter ce qu'est le deep learning et ce que sont les réseaux de neurones, et surtout la différence entre les différentes méthodes de classification d'images.
\section{Géneralité}
    \subsection{Definition}\textbf{Deep Learning (DL):} 
            
        Selon Wikipedia L'apprentissage profond ou apprentissage en profondeur est un ensemble de méthodes d'apprentissage automatique tentant de modéliser avec un haut niveau d’abstraction des données grâce à des architectures articulées de différentes transformations non linéaires. Ces techniques ont permis des progrès importants et rapides dans les domaines de l'analyse du signal sonore ou visuel et notamment de la reconnaissance faciale, de la reconnaissance vocale, de la vision par ordinateur, du traitement automatisé du langage. Dans les années 2000, ces progrès ont suscité des investissements privés, universitaires et publics importants, notamment de la part des GAFAM (Google, Apple, Facebook, Amazon, Microsoft).
    \subsection{Historique}

    \begin{figure}[H]
        \centering
        \includegraphics[width=0.8\textwidth]{history-of-ai.jpg}
        \caption{Chronologie du développement et de l'utilisation de l'intelligence artificielle en médecine. 
        À L' intelligence artificielle ; ID, apprentissage profond ; FDA, Agence fédérale américaine des produits alimentaires et médicamenteux  : CAO, diagnostic assisté par ordinateur.
        }
        \label{fig:history-of-ai}
    \end{figure}

\section{Réseau des neurones}
        un réseau de neurones est une structure qui aide l'ordinateur à imiter le processus d'apprentissage du cerveau humain, en collectant un certain nombre de nœuds liés par des poids,la manifestation la plus élémentaire d'un réseau de neurones est ce qu'on appelle le perceptron, il s'agit d'un réseau de neurones à une seule couche composé de deux entrées et d'une sortie comme suit:

        \subsection{Perceptron}
            Comme mentionné ci-dessus, le perceptron est  un lien neuronal à une seule couche avec quatre paramètres principaux. Le modèle perceptron multiplie d'abord toutes les valeurs d'entrée et leurs poids, puis ajoute ces valeurs pour créer une somme pondérée. En outre, cette somme pondérée est appliquée à la fonction d'activation "f" pour obtenir la sortie souhaitée. Cette fonction d'activation est également appelée fonction échelon et est désignée par la lettre "f".
            \begin{figure}[H]
                \centering
                \includegraphics[scale=0.3]{perceptron}
                \caption{Perceptron}
                \label{fig:perceptron}
            \end{figure}

            \begin{itemize}[label=$\bullet$]
                \item Xi: entrées.
                \item Wi: poids.
                \item f: fonction d'activation.
                \item y: sortie 
            \end{itemize}

            cette fonction d'activation est nécessaire pour s'assurer que la sortie est  mappée entre (0,1) ou (-1,1). Notez que le poids de l'entrée indique la force d'un nœud. De même, une valeur d'entrée donne la possibilité de décaler la courbe de la fonction d'activation vers le haut ou vers le bas.
            
            Étape 1: Calculez la somme pondérée en multipliant toutes les valeurs d'entrée par leurs valeurs de poids respectives et en les additionnant. La formule est:

            \begin{equation}\label{eq:per_sum}
                \Sigma = \Sigma wi*xi = w1*x1 + w2*x2 + w3*x3 + \dots
            \end{equation}
            \myequations{Somme pondérée des entrées par leur poids}


            Nous ajoutons un terme appelé biais « b » à cette somme pondérée pour améliorer les performances du modèle. 
            Étape 2: La fonction d'activation est appliquée à l'aide des sommes pondérées ci-dessus pour donner la sortie sous forme binaire ou des valeurs continues comme suit:


            \begin{equation}\label{eq:per_func}
                Y = f(\Sigma + b)
            \end{equation}
            \myequations{Calcule de sortie}

            Étape 3: finalement pour améliorer la précision du perceptron il faut ajuster les poids suivants la relation:

            \begin{equation}\label{eq:per_weights}
                W_n = W_a + \alpha (y_r - Y) X
            \end{equation}
            \myequations{Equation de calcule des nouveaux poids}
            

            Avec \boldmath{\(W_n\)} le nouveau vecteur des poids, \textbf{\(W_a\)} le dernier vecteur de poids, \textbf{\(\alpha\)} est le pas d'apprentissage, \textbf{\(y_r\)} la sortie attendée (réele), \textbf{Y} est la sortie prévue par le perceptron et finalement \textbf{X} est le vecteur contenant les valeur courants des entrée.

        En bref, un réseau neuronal profond est une structure compliquée composée de plusieurs perceptrons interconnectés. Dans un réseau neuronal, nous avons trois unités principales, la couche d'entrée, la couche de sortie et la couche cachée contenant plusieurs couches interconnectées.
        
        \begin{figure}[H] 
            \centering
            \includegraphics[width=0.8\textwidth]{multi_layer_perce.png}
            \caption{multi couche perceptron}
            \label{fig:m_l_p}
        \end{figure}

    \subsection{Réseau de neurones convolutionnel (RNC)}
    En apprentissage automatique, un réseau de neurones convolutifs (CNN ou ConvNet) est un type de réseau de neurones artificiels acycliques (feedforward) dont les schémas de connectivité entre les neurones sont inspirés du cortex visuel des animaux. Les neurones de cette zone du cerveau sont disposés pour correspondre à des zones qui se chevauchent lorsque le champ visuel est carrelé. Leur fonctionnalité s'inspire des processus biologiques et consiste en un empilement multicouche de perceptrons destiné à prétraiter de petites quantités d'informations. Les réseaux de neurones convolutifs ont un large éventail d'applications, notamment la reconnaissance d'images et de vidéos, les systèmes de recommandation et le traitement du langage naturel.

        \subsubsection{Architecture RNC standard}
        La forme la plus courante d'architecture de réseau neuronal convolutif empile plusieurs couches Conv-ReLU, suivies de couches de pool supplémentaires, répétant ce modèle jusqu'à ce que l'entrée soit réduite à un espace  suffisamment petit. À un moment donné, il est courant de mettre une couche entièrement connectée (FC). La dernière couche entièrement connectée sera connectée à la sortie. Vous trouverez ci-dessous une architecture de réseau neuronal convolutif commune qui suit ce modèle:

        \begin{figure}[H] 
            \centering
            \includegraphics[width=\textwidth]{conv_stand_arch.png}
            \caption{Architecture standard d'un Réseau de neurones convolutionnel}
            \label{fig:conv_stand_arch}
        \end{figure}

        \subsubsection{Couche convolutive}
        Les couches convolutives sont à la base des CNN. Un paramètre de tranche consiste en un ensemble de filtres d'apprentissage (ou noyaux). Ces filtres ont un petit champ de vision, mais couvrent toute la profondeur du volume d'entrée. Dans le (Feed forward), chaque filtre est convolué sur la largeur et la hauteur du volume d'entrée et le produit scalaire entre les entrées du filtre est calculé pour produire la carte d'activation 2D pour ce filtre. Par conséquent, le réseau apprend quels filtres s'activent lorsqu'il détecte certains types d'entités à certains emplacements spatiaux dans l'entrée. 
        L'empilement des cartes d'activation de tous les filtres le long de la dimension de profondeur forme le volume de sortie complet de la couche de convolution. Chaque entrée dans le volume de sortie peut donc également être interprétée comme la sortie d'un neurone partageant des paramètres avec des neurones dans la même carte d'activation en regardant une petite région à l'intérieur de l'entrée.

        \subsubsection{Couche max pooling}
        Une couche max pooling est une nouvelle couche ajoutée après la couche de convolution. Surtout après avoir appliqué des non-linéarités (telles que ReLU)  aux cartes de caractéristiques générées par les couches convolutionnelles. 
        L'ajout d'une couche de max pooling après une couche convolutive est un modèle courant utilisé pour ordonner les couches dans les réseaux de neurones convolutifs qui se répètent une ou plusieurs fois dans un modèle particulier. 
        La couche max pooling fonctionne sur chaque carte d'entités indépendamment, créant un nouvel ensemble du même nombre de cartes d'entités regroupées.

        
        \subsubsection{Hyperparamètres}
        En apprentissage automatique, les hyperparamètres sont des paramètres dont les valeurs sont utilisées pour contrôler le processus d'apprentissage. En revanche, les valeurs des autres paramètres (généralement les poids des nœuds) sont obtenues par apprentissage. 
        Les hyperparamètres peuvent être classés comme des hyperparamètres de modèle qui ne peuvent pas être dérivés en ajustant une machine à un ensemble d'apprentissage, ou des hyperparamètres de modèle, car ils s'appliquent à la tâche de sélection de  modèle. Algorithmes qui, en principe, n'affectent pas les performances du modèle, mais affectent la vitesse et la qualité du processus d'apprentissage. Des exemples d'hyperparamètres de modèle sont la topologie et la taille du réseau neuronal. Des exemples d'hyperparamètres algorithmiques sont le taux d'apprentissage et la taille de la pile.
        \paragraph{Choix des hyperparamètres}
        Les réseaux de neurones convolutifs utilisent plus d'hyperparamètres que les perceptrons multicouches standards. Même si les règles habituelles de taux d'apprentissage et de constantes de régularisation s'appliquent toujours, les notions de nombre de filtres, leur forme et la forme du max pooling doivent être prises en considération.




\section{Classification des images}
    La classification des images est une tâche fondamentale qui tente de comprendre l'image dans son ensemble. Le but est de classer les images en leur attribuant des étiquettes spécifiques. La classification des images fait généralement référence aux images dans lesquelles un seul objet est affiché et  analysé. En revanche, la détection d'objets comprend à la fois des tâches de classification et de localisation et est utilisée pour analyser des cas plus réalistes où plusieurs objets peuvent être présents dans une image.

    Dans le domaine de la classification des images, nous avons exploré trois types bien connus:
    \begin{enumerate}
        \item Classification binaire
        \item Classification Multi-Classes
        \item Classification Multi-Etiquettes
    \end{enumerate}

    \subsection{Classification binaire}
        La classification binaire fait référence aux tâches de classification avec deux étiquettes de classe. 
        Les tâches de classification binaire ont généralement une classe représentant les conditions normales et une autre classe représentant les conditions anormales. 
        Et dans notre cas, dans l'image, on fait référence à la présence d'objets (en prenant l'étiquette 0) ou non (en prenant l'étiquette 1).
    \subsection{Classification Multi-Classes}
        Contrairement à la classification binaire, la classification multi-classes n'a pas la notion de résultats normaux et anormaux. Au lieu de cela, les exemples sont classés comme appartenant à l'une parmi une gamme de classes connues.

        Le nombre d'étiquettes de classe peut être très important sur certains problèmes. Par exemple, un modèle peut prédire qu'une photo appartient à un parmi des milliers ou des dizaines de milliers de visages dans un système de reconnaissance faciale.

        Mais la chose la plus importante à noter est que le résultat prédit de ce modèle est une classe unique qui le rend moins intéressant dans notre cas où une radiographie peut avoir de multiples anomalies.
    \subsection{Classification Multi-Etiquettes}
        Ceci est différent de la classification binaire ou multiclasse, où une seule étiquette de classe est prédite par instance. 
        Il est courant de modéliser des tâches de classification multi-étiquettes à l'aide de modèles qui prédisent plusieurs sorties, où chaque sortie est prédite sous la forme d'une distribution de probabilité de Bernoulli. Il s'agit essentiellement d'un modèle qui effectue plusieurs prédictions de classification binaire par exemple. 
        Les algorithmes de classification utilisés pour la classification binaire ou multiclasse ne peuvent pas être utilisés directement pour la classification multiétiquette. Une version spéciale de l'algorithme de classification standard est disponible, appelée version multi-étiquettes de l'algorithme:
        \begin{itemize}[label=$\bullet$]
            \item Arbres de décision multi-étiquettes
            \item Forêts aléatoires multi-étiquettes
            \item Amplification des dégradés multi-étiquettes
        \end{itemize}


\section{Apprentissage par transfert}
    L'apprentissage par transfert est la réutilisation de modèles pré-entraînés pour de nouveaux problèmes. Il est actuellement très populaire  en apprentissage profond car il permet d'entraîner des réseaux de neurones profonds avec relativement peu de données. Ceci est très utile dans le domaine de la science des données. En effet, la plupart des problèmes du monde réel n'ont généralement pas des millions de points de données étiquetés pour entraîner un modèle aussi complexe.


\section*{Conclusion}
        Ce qu'on peut en deduire de ce chapitre c'est que pour la réussite de notre projet on doit se baser sur 3 concept principales:
        \begin{enumerate}
            \item Réseaux de neurones convolutifs
            Pour le traitement des données qui ont sous forme de clichés de radiologiques (images)
            \item Classification Multi-Etiquettes
            Car on veut que le modèle détecte tout les anomalies possible non seulement une.
            \item Apprentissage par transfert
            On a pas assés de données pour crée un modèle puissant donc on va se baser sur un modèle pré-entraîné, pour assurer une bon précision avec le minimum de données.
        \end{enumerate}
    \chapter{Etude technique}

\section*{Introduction}
La conception et la modélisation sont à la base du projet. Les connaître nous permettra de bien définir les autres composantes et besoins de ce projet. Cela comprend les choix techniques et les outils informatiques utilisés pour obtenir des résultats satisfaisants. C'est un projet  structuré avec des composants interconnectés pour un excellent flux de travail.

\section{Conception et modelisation}
Dans tout ce qui précède, nous pouvons remarquer une structure légèrement différente de celle que nous avons déjà montrée dans la section \ref{Solution_prop}.
    \subsection{Structure générale du projet}
    Pour construire une nouvelle structure, il faut s'appuyer sur les points issus du chapitre précédent.

    Pour ce faire, on doit utiliser une base de données secondaire. Il s'agit d'une grande base de données structurée de la population adulte, pour construire des modéles pré-entraînés sur des donnéees de la population adulte, Un phénomène décidément très proche du nôtre.
    Après tous ces considérations on a trouver la structure du figure \ref{fig:structure_gen}
    \begin{figure}[H]
        \centering
        \includegraphics[width=0.7\textwidth]{structure_gen.jpg}
        \caption{structure génerale du projet}\label{fig:structure_gen}
    \end{figure}
    \subsection{Application Web}
        L'application Web est une API simple pour la collection et la visualisation de données, donc son architecture de base est également simple, c'est comme suit figure \ref{fig:xpedia_arc}
        \begin{figure}[H]
            \centering
            \includegraphics[width=1\textwidth]{xpedia_arc.png}
            \caption{structure génerale de l'application web Xpedia}\label{fig:xpedia_arc}
        \end{figure}
        \subsubsection{UML}
            UML (abréviation de Unified Modeling Language) est un langage de modélisation standardisé composé d'un ensemble de diagrammes intégrés qui aident les développeurs de systèmes et de logiciels à spécifier, visualiser et visualiser les artefacts des systèmes logiciels. .système. UML représente un ensemble éprouvé de meilleures pratiques techniques pour la modélisation de systèmes vastes et complexes et constitue une partie très importante du développement logiciel orienté objet et du processus de développement logiciel. UML utilise principalement une notation graphique pour représenter la conception de projets logiciels.

            \begin{enumerate}
                \item Diagramme de classe:
                Notre diagramme de classe de  deux classes utilisateur et xray (cliché), notre but de cette application et de collectés les données (xrays), l'acteur qui va effectuer la collection est l'utilisateur:
                \begin{figure}[H]
                    \centering
                    \includegraphics[width=0.8\textwidth]{class_dia.png}
                    \caption{Le diagramme de classe de l'application web Xpedia}\label{fig:class_dia}
                \end{figure}
                \item Diagrammes de séquence
                \begin{enumerate}
                    \item Diagramme de séquence: cas d'authentification
                    \begin{figure}[H]
                        \centering
                        \includegraphics[width=0.5\textwidth]{cas_auth.png}
                        \caption{Le diagramme de séquence: cas d'authentification}\label{fig:cas_auth}
                    \end{figure}

                    Comme on peut le voir sur la figure \ref{fig:cas_auth}, l'authentification se fait à travers 2 opérations principales:
                    \begin{itemize}[label=$\bullet$]
                        \item l'utilisateur fournit le nom d'utilisateur et le mot de passe
                        \item le système recoupe les informations fournies avec les données existantes dans la collection de l'utilisateur
                    \end{itemize}
                    \item Diagramme de séquence: cas d'ajouter un cliché
                    \begin{figure}[H]
                        \centering
                        \includegraphics[width=0.5\textwidth]{cas_add.png}
                        \caption{Le diagramme de séquence: cas d'ajouter un cliché}\label{fig:cas_add}
                    \end{figure}
                \end{enumerate}
            \end{enumerate}
        
    \subsection{Pipeline de données}
    \paragraph*{Un pipeline de données est} une série d'étapes de traitement de données. Si les données ne sont pas actuellement chargées dans la plateforme de données, elles seront ingérées au début du pipeline. Ensuite, vous avez une séquence d'étapes où chaque étape fournit une sortie qui devient l'entrée de l'étape suivante. Cela continue jusqu'à ce que le pipeline soit terminé. Dans certains cas, des étapes indépendantes peuvent s'exécuter en parallèle.  
    
    Un pipeline de donnéesse compose de trois éléments principaux : une source, une ou plusieurs étapes de traitement et une destination. Dans certains pipelines de données, une destination peut être appelée un récepteur. Par exemple, un pipeline de données peut transmettre des données de votre application à un entrepôt de données, d'un lac de données à une base de données d'analyse ou à un système de traitement des paiements. Un pipeline de données peut également avoir la même source et le même récepteur, de sorte que le pipeline n'a besoin que de modifier l'ensemble de données. Chaque fois que des données sont traitées entre le point A et le point B (ou les points B, C, D), il existe un pipeline de données entre ces points.

        \subsubsection{Integration}
        L'intégration de données est le processus consistant à combiner des données provenant de différentes sources en une vue unique et unifiée. L'intégration commence par le processus d'ingestion et comprend des étapes telles que le nettoyage, le mappage ETL et la transformation. L'intégration des données permet finalement aux outils d'analyse de produire une intelligence économique efficace et exploitable.

        Il n'existe pas d'approche universelle de l'intégration des données. Cependant, les solutions d'intégration de données impliquent généralement quelques éléments communs, notamment un réseau de sources de données, un serveur maître et des clients accédant aux données à partir du serveur maître.

        Dans un processus d'intégration de données typique, le client envoie une demande de données au serveur maître. Le serveur maître reçoit ensuite les données nécessaires à partir de sources internes et externes. Les données sont extraites des sources, puis consolidées en un seul ensemble de données cohérent. Ceci est renvoyé au client pour utilisation.
        \subsubsection{Transformation}
        La transformation des données est le processus de transformation des données d'un format à un autre. Généralement, vous convertissez du format du système source au format requis par le système cible. Les transformations de  données font partie de la plupart des tâches d'intégration et de gestion des données, telles que : B. Gestion des données et stockage des données. 

        En tant qu'étape du processus ELT/ETL, la transformation des données peut être qualifiée de « simple » ou de « complexe » selon le type de modifications qui doivent être apportées aux données avant qu'elles ne soient livrées à leur destination. Le processus de transformation des données peut être effectué à l'aide de l'automatisation, de l'administration manuelle ou d'une combinaison des deux.

        \subsubsection{Réduction}
        La réduction des données signifie la réduction de certains aspects des données, généralement le volume de données. La réduction peut également porter sur d'autres aspects tels que la dimensionnalité des données lorsque les données sont multidimensionnelles. La réduction de tout aspect des données implique généralement une réduction du volume de données.

        La réduction des données n'a de sens en soi que si elle est associée à une certaine finalité. Le but à son tour dicte les exigences pour les techniques de réduction de données correspondantes. Un but naïf de la réduction des données est de réduire l'espace de stockage. Cela nécessite une technique pour compresser les données dans un format plus compact et également pour restaurer les données d'origine lorsque les données doivent être examinées. De nos jours, l'espace de stockage n'est peut-être pas la principale préoccupation et les besoins de réduction des données proviennent fréquemment des applications de base de données. Dans ce cas, le but de la réduction des données est d'économiser le coût de calcul ou le coût d'accès au disque dans le traitement des requêtes.
        \subsubsection{Nettoyage}
        Le nettoyage des données est le processus de réparation ou de suppression des données incorrectes, corrompues, mal formées, en double ou incomplètes dans un ensemble de données. La combinaison de plusieurs sources de données augmente le risque de dupliquer ou de mal étiqueter les données. Si les données sont erronées, même si elles semblent correctes, les résultats et les algorithmes ne seront pas fiables. Il n'existe aucun moyen absolu de dicter les étapes exactes du processus de nettoyage des données, car le processus est différent pour chaque ensemble de données. Cependant, il est important de définir un modèle pour votre processus de nettoyage des données et de vous assurer que vous le faites correctement à chaque fois.


   
    \subsection{Modèles de deeplearning}

    \begin{figure}[H]
        \centering
        \includegraphics[width=1\textwidth]{adulte_model_train.png}
        \caption{La schéma de créeation du modèle des radiographies adultes}\label{fig:adulte_model_schema}
    \end{figure}
    \begin{figure}[H]
        \centering
        \includegraphics[width=1\textwidth]{pedia_model_train.png}
        \caption{La schéma de créeation du modèle des radiographies adultes}\label{fig:pedia_model_schema}
    \end{figure}
\section{Choix techniques}
    choix techniques
    \subsection{Architecture logicielle}

    \subsection{Architecture des réseaux de neurones}

\section{Bases de données}
    \subsection{CheXpertDB}\label{chexpertDB}
    \subsection{XpediaDB}

\section*{Conclusion}
    \chapter{Réalisation}

\section*{Introduction}

\section{Réalisation}

\section{Experiences}

\section{Résultats}

\section*{Conclusion}

    \chapter*{Conclusion}
    Notre projet s'articule autour de l'apport de l'intelligence artificielle à l'interprétation d'images radiographiques thoraciques, notamment pour une population pédiatrique sur 13 pathologies prédéfinies.
    
    Le manque de données a été principalement résolu avec la collecte de données, via l'interface de l'application Web Xpedia, en utilisant les ressources de l'UHC, y compris les équipements et un personnel de résidents, en utilisant également un ensemble de données adulte existant pour profiter des fonctionnalités d'apprentissage par transfert.
    
    La préparation des données a pris une portion importante et nécessaire du temps alloué aux projets, afin d'être entièrement préparé pour la phase de formation.

    La construction de modèles d'apprentissage en profondeur s'est basée sur de nombreuses sessions de recherche et de lecture pour avoir une bonne idée des tenants et des aboutissants de certains des concepts d'apprentissage en profondeur entièrement développée, afin de les exploiter pleinement.

    La phase d'apprentissage a également pris beaucoup de temps et de recherches, en particulier pour trouver des ressources externes alternatives pour exécuter nos scripts, ce qui nous a guidés vers HPC-MARWAN.

    Les résultats valident nos hypothèses concernant la viabilité de la méthode et ils nous ont fourni des informations de base pour adapté et amélioré le modèle.

    Ce fut un grand voyage explorant les technologies d'apprentissage en profondeur dans les avantages de la médecine J'ai certainement beaucoup appris et il y a beaucoup plus à apprendre tout au long des phases et expériences futures de ce projet
    \clearpage

    \chapter*{Références}
    [01]     Yoo et al., « AI-Based Improvement in Lung Cancer Detection on Chest Radiographs »;

    [02]     Rangarajan et al., « Artificial Intelligence–assisted chest X-ray assessment scheme for COVID-19 »;
   
    [03]     Tang et al., « Automated Abnormality Classification of Chest Radiographs Using Deep Convolutional Neural Networks »;
   
    [04]     O’Brien et al., « Causes of Severe Pneumonia Requiring Hospital Admission in Children without HIV Infection from Africa and Asia »;
   
    [05]     Chen et al., « Deep Learning for Classification of Pediatric Chest Radiographs by WHO’s Standardized Methodology »;
   
    [06]     « History of Artificial Intelligence in Medicine | Elsevier Enhanced Reader »; Kermany et al., « Identifying Medical Diagnoses and Treatable Diseases by Image-Based Deep Learning »;
   
    [07]     E. Blondiaux, C. de Labriolle-Vaylet (Nom), « Imagerie pédiatrique »; « Intelligence Artificielle et Imagerie Médicale »;
   
    [08]     « L’intelligence artificielle en médecine | IBM »; Anichini et Geffroy, « L’intelligence artificielle à l’épreuve des savoirs tacites. Analyse des pratiques d’utilisation d’un outil d’aide à la détection en radiologie », 2021;
   
    [09]     Anichini et Geffroy, « L’intelligence artificielle à l’épreuve des savoirs tacites. Analyse des pratiques d’utilisation d’un outil d’aide à la détection en radiologie », 2021;
   
    [10]    Padash et al., « Pediatric chest radiograph interpretation »;
   
    [11]    Tang, « The role of artificial intelligence in medical imaging research »;
    
    [12]    Rahman et al., « Transfer Learning with Deep Convolutional Neural Network (CNN) for Pneumonia Detection Using Chest X-Ray »;
    
    [13]    Qin et al., « Tuberculosis Detection from Chest X-Rays for Triaging in a High Tuberculosis-Burden Setting »;
    
    [14]    Nguyen et al., « Viral and Bacterial Pneumonia Diagnosis via Deep Learning Techniques and Model Explainability ».

    [15]    V.M. Corman et al., Detection of 2019 novel coronavirus (2019-nCoV) by real-time RT-PCR. Euro. Surveill. 25(3), 2000045 (2020).
    
    [16]    Y. Fang et al., Sensitivity of Chest CT for COVID-19: comparison to RT-PCR. Radiology 296(2), E115–E117 (2020).

    [17]    T.C. Williams et al., Sensitivity of RT-PCR testing of upper respiratory tract samples for SARS-CoV-2 in hospitalised patients: a retrospective cohort study. medRxiv (2020).

    [18]    J.P. Kanne, B.P. Little, J.H. Chung, B.M. Elicker, L.H. Ketai, Essentials for radiologists on COVID-19: An Update—Radiology Scientific Expert Panel. Radiology 296(2), E113–E114 (2020).

    [19]    J.F.-W. Chan et al., A familial cluster of pneumonia associated with the 2019 novel coronavirus indicating person-to-person transmission: a study of a family cluster. Lancet 395(10223), 514–523 (2020).

    [20]    L.A. Rousan, E. Elobeid, M. Karrar, Y. Khader, Chest x-ray findings and temporal lung changes in patients with COVID-19 pneumonia. BMC Pulmonary Med. 20(1), 245 (2020).

    [21]    S. Asif, Y. Wenhui, H. Jin, Y. Tao, S. Jinhai, Classification of COVID-19 from chest X-ray images using deep convolutional neural networks. medRxiv (2020).

    [22]    S. Anis et al., An overview of deep learning approaches in chest radiograph. IEEE Access 8, 182347–182354 (2020).

    [23]    A. Shelke et al., Chest X-ray classification using deep learning for automated COVID-19 screening. medRxiv (2020).

    [24]    A. K. Das, S. Ghosh, S. Thunder, R. Dutta, S. Agarwal, and A. Chakrabarti, "Automatic COVID-19 detection from X-ray images using ensemble learning with convolutional neural network," Pattern Analysis and Applications. 2021/03/19 2021. 

    \subsection*{Wikipedia: }
    \url{https://en.wikipedia.org/wiki/Deep_learning}

    \url{https://en.wikipedia.org/wiki/Transfer_learning}
    \subsection*{Towards Data Science:}
    \url{https://towardsdatascience.com/understanding-variational-autoencoders-vaes-f70510919f73}

    \url{https://towardsdatascience.com/understanding-binary-cross-entropy-log-loss-a-visual-explanation-a3ac6025181a}

    \url{https://towardsdatascience.com/feature-extraction-techniques-d619b56e31be}
    \subsection*{Medium:}
    \url{https://medium.com/@iamarjunchandra/mixed-input-data-in-pytorch-cnn-mlp-8aeff336e8a3}

    \url{https://medium.com/@mehulved1503/feature-selection-and-feature-extraction-in-machine-learning-an-overview-57891c595e96}

    \subsection*{Machine Learning Mastery:}
    \url{https://machinelearningmastery.com/transfer-learning-for-deep-learning/}

    \url{https://machinelearningmastery.com/convolutional-layers-for-deep-learning-neural-networks/}

    \subsection*{GitHub:}
    \url{https://github.com/poloclub/cnn-explainer}

    \url{https://github.com/topics/variational-autoencoders}

    \subsection{Keras:}
    \url{https://keras.io/api/layers/convolution_layers/convolution2d/}

    \url{https://keras.io/api/layers/activations/#relu-function}

    \url{https://keras.io/api/layers/activations/#sigmoid-function}

    \url{https://keras.io/api/layers/core_layers/dense/}

    \url{https://keras.io/api/layers/pooling_layers/max_pooling2d/}

    \url{https://keras.io/api/layers/normalization_layers/batch_normalization/}

    \url{https://keras.io/api/layers/regularization_layers/dropout/}

    \subsection{Autres: }

    \url{https://www.mongodb.com/mern-stack}

    \url{https://beta.reactjs.org/}

    \chapter*{Annexe}\label{Annex}
    \section{Réalisation}
    \subsection{Application web}
    \begin{figure}[H]
        \centering
        \includegraphics[width=0.5\textwidth]{xpedia_log.png}
        \caption{La page login de Xpedia}\label{fig:xpedia_log}
    \end{figure}
    \begin{figure}[H]
        \centering
        \includegraphics[width=\textwidth]{xpedia_dashboard.png}
        \caption{Le Dashboard}\label{fig:xpedia_dashboard}
    \end{figure}
    \begin{figure}[H]
        \centering
        \includegraphics[width=1\textwidth]{xpedia_menu.png}
        \caption{Le menu}\label{fig:xpedia_menu}
    \end{figure}
    \begin{figure}[H]
        \centering
        \includegraphics[width=0.4\textwidth]{xpedia_select_section.png}
        \caption{La section de sélection des diagnostics}\label{fig:xpedia_select_section}
    \end{figure}
    \begin{figure}[H]
        \centering
        \includegraphics[width=0.4\textwidth]{xpedia_select_items.png}
        \caption{Exemple de sélection des diagnostics}\label{fig:xpedia_select_items}
    \end{figure}
    \begin{figure}[H]
        \centering
        \includegraphics[width=\textwidth]{xpedia_cr.png}
        \caption{La section du remplissage du compte rendu}\label{fig:xpedia_cr}
    \end{figure}
    \begin{figure}[H]
        \centering
        \includegraphics[width=\textwidth]{xpedia_browse_item.png}
        \caption{La page ’browse items’}\label{fig:xpedia_menu}
    \end{figure}
    \begin{figure}[H]
        \centering
        \includegraphics[width=0.4\textwidth]{xpedia_filter_section.png}
        \caption{La section du filtre des éléments}\label{fig:xpedia_filter_section}
    \end{figure}
    \begin{figure}[H]
        \centering
        \includegraphics[width=\textwidth]{xpedia_pagination.png}
        \caption{La section pagination}\label{fig:xpedia_menu}
    \end{figure}
    
    \begin{figure}[H]
        \centering
        \includegraphics[width=1\textwidth]{xpedia_view_page.png}
        \caption{Les détaille de l’élément}\label{fig:xpedia_view_page}
    \end{figure}
    
    \begin{figure}[H]
        \centering
        \includegraphics[width=\textwidth]{xpedia_edit_page.png}
        \caption{La page d’édition de l’élément}\label{fig:xpedia_item_thumbnail}
    \end{figure}

    \url{https://github.com/GharafiHamza/xpedia_db_cleaning}\label{repo}
    

\end{document}